\section*{Sammendrag}
\addcontentsline{toc}{section}{Sammendrag}
Denne forskningsoppgaven tar for seg elever og lærere ved Porsgrunn videregående skole sine holdninger til personvern og overvåkning på nett. Med denne oppgaven ønsker jeg å få en forståelse og oversikt over holdningene og kunnskapen til elevene og lærerne ved Porsgrunn videregående skole som personvern og overvåkning. 

Dette skal undersøkes i lys av tre temaer: elevene og lærernes kunnskap til lovverk til personvern, deres kunnskap og holdning om overvåkning fra offentlige og private aktører, og deres kunnskap og holdning til overvåkning mellom privatpersoner.

Ved hjelp av en anonym undersøkelse skal elevene og lærerne formidle deres kunnskap og holdninger om temaene.

\section*{Forord}
\addcontentsline{toc}{section}{Forord}
Å skrive denne forskningsoppgaven har vært mer krevende en jeg forventet, og har vert en prosess over lang tid. Mye innhold har blitt endret underveis, og spørreundersøkelsen, som har vært en stor del av oppgaven, tok lang tid å både formulere riktig, og få nok deltagelse på. Jeg vil derfor takke veilederen min Margareth Rimmereide som har fulgt opp og vurdert underveis, og veiledet meg gjennom prosessen. 

Jeg vil også rette en stor takk til bibliotekaren ved avdeling Nord, Maria Terzan, som hjalp til med markedsføringen av undersøkelsen ved å informere alle elevene og lærerne gjennom et fellesrom på Teams, og gi meg tillatelse til å henge opp plakater ved biblioteket.

Til slutt vil jeg takke alle de 77 elevene og lærerne som stilte som informanter og svarte på undersøkelsen, uten deres svar kunne ikke denne oppgaven vært mulig.

\vfill

Håvard Solberg Nybøe

\today{}


\newpage