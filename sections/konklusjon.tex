\section{Konklusjon}

\subsection{Oppsummering}
I løpet av oppgaven har det kommet fram at én av hypotesene som ble lagt fram i kapittel \ref{subsubsec:hypoteser} stemte, og én var feil. Problemstillingen fikk et endelig svar, på bakgrunn av resultatene i undersøkelsen. I tillegg håper jeg at noen av informantene som deltok i undersøkelsen ble mer bevisst på personvern og kanskje begynte å tenke mer på hva de etterlater seg av spor og informasjon på nett.

Hensikten med oppgaven var å få en oversikt over personopplysninger som blir lagret om oss, og finne ut hva de ble brukt til. Jeg føler at hensikten har blitt oppnådd til den grad det var mulig i oppgaven, med tanke på området og temaet oppgaven tok for seg. Oppgaven skulle også gi svar på hva elevene og lærerne ved Porsgrunn videregående skole visste om personvern, hvilke av deres personopplysninger som ble lagret, og hva de ble brukt til. Undersøkelsen var hovedmetoden som skulle ta for seg dette, og det føler jeg den greide godt nok.

\subsection{Veien videre}
Det var flere deler ved oppgaven jeg gjerne skulle gått dypere inn på, men som ikke passet inn heller ville tatt for stor plass i den begrensede mengden jeg skulle ta for meg. Noe av det som tidlig var planlagt å ta med, men som like tidlig måtte strykes, var å laste ned reell oppsamlet data fra ulike bedrifter og analysere de. Det hadde vert spennende å finne ut av hva slags data ulike aktører samlet opp og ikke minst hvor mye de lagret og visste om oss.

Jeg ville også gjerne undersøkt mer om sikkerheten runt håndtering av personopplysninger, og elevene og lærernes kjennskap til det. Med det mener jeg hvor de lagres (lokale vs. eksterne servere) og hvordan de lagres, altså ulik kryptering og av dataen (``hashing'' og kryptering). Det hadde vært spennende å finne ut av om almenheten viste at dataene blir kryptert eller om de trodde den bare ble lagret i ren tekst.

\newpage
