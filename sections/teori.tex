\section{Teori om temaet}

\subsection{Hva er personvern?}
Å ha en riktig definisjon på personvern er viktig for å ha en korrekt forståelse av hva personvern og personopplysninger er. Så å definere personvern er en oppgave for de som jobber med det til daglig, som for eksempel regjeringen og Datatilsynet. De har henholdsvis definert personvern slik:

\textit{``Personvern er derfor nært knyttet til enkeltindividers muligheter for privatliv, selvbestemmelse og selvutfoldelse.''} \parencite{artikkel:regjeringen_personvern}

\textit{``Personvern handler om retten til et privatliv og retten til å bestemme over egne personopplysninger.''} \parencite{artikkel:datatilsynet_personvern}

Begge disse definisjonene sier omtrent det samme, og legger fram at personvern er nært knyttet til privatliv og selvbestemmelse over egne personopplysninger.

\subsection{Hva er GDPR?}
\textit{The General Data Protection Regulation}, også kjent som GDPR, er i følge EU ``\emph{the toughest privacy and security law in the world}''. Loven trådte i kraft 25. mai 2018, og har som formål å styrke personvernet til innbyggere i EU ved å gjøre brukerne mer opplyst over hva slags data som blir samlet inn, og gjøre det enkelt å laste ned eller slette dataen. Selv om loven ble utarbeidet og trådt i kraft av EU gjelder den alle organisasjoner uavhengig av landstilhørighet, så lenge de behandler eller samler opp data relatert til innbyggere av EU. Dette gjør at loven også gjelder for land som ikke med medlem av EU, som for eksempel Norge, eller land som ikke ligger i Europa som for eksempel USA. \parencite{artikkel:eu_gdpr}

I tillegg ble det i Norge, to måneder etter at GDPR trådte i kraft, vedtatt og trådt i kraft en ny personopplysningslov som gjennomfører personvernforordningen fra EU, altså GDPR, i Norge. Dette vil si at GDPR loven da gjelder i Norge uavhengig av den som er gitt sentralt fra EU. \parencite{artikkel:regjeringen_gdpr}

\subsection{Den nye etterretningstjenesteloven i Norge}
Den 11. juni 2020 vedtok Stortinget en ny lov om etterretningstjenesten. \parencite{artikkel:regjeringen_nylov}
Lovens formål er å ``\textit{bidra til å trygge Norges suverenitet, territorielle integritet, demokratiske styreform og andre nasjonale sikkerhetsinteresser, herunder forebygge, avdekke og motvirke utenlandske trusler mot Norge og norske interesser; bidra til å trygge tilliten til og sikre grunnlaget for kontroll med Etterretningstjenestens virksomhet; sikre at Etterretningstjenestens virksomhet utøves i samsvar med menneskerettighetene og andre grunnleggende verdier i et demokratisk samfunn.}'' \parencite[§1-1]{artikkel:lovdata_etterretningstjenesten} 
\newpage

Med dette menes det at loven skal bedre sikre Norge mot eventuell nettbasert kriminalitet eller terror, og at disse inngrepene skal skje i samsvar med menneskerettighetene og andre grunnleggende verdier. Loven trådte i kraft 1. januar 2021.

En av metodene for å hindre kriminalitet, som denne loven innfører, er lagring av metadata. \parencite[§7-7]{artikkel:lovdata_etterretningstjenesten} Metadata er kort fortalt data som inneholder informasjon som beskriver annen data; det er data om data. Til et bilde kan metadata for eksempel være dataen som forteller når bilder ble tatt, bildets oppløsning og dimensjoner eller filtypen til bildet. 

Metadataen som samles opp skal etter loven slettes etter 18 måneder, og kan kun hentes opp av ``\textit{personell i Etterretningstjenesten som er vurdert som skikket til det og som utpekes av sjefen for tjenesten}'' \parencite[§7-8]{artikkel:lovdata_etterretningstjenesten}

Denne loven fikk raskt kritikk av blant annet Datatilsynet som i 2019 skrev en artikkel om deres holdning til loven, etter at Forsvarsdepartementet sende ut et forslag om loven i november 2018. Artikkelen går ut på at Datatilsynet mener denne nye loven, hvor staten har lov til å samle opp metadata om norske innbyggere, er et for stort inngrep på retten til privatliv og kaller det for en ``\textit{digital masseovervåking av norske borgere}''. Datatilsynet skriver også at de mener denne loven går i strid med Norges grunnlov og menneskerettighetene. \parencite{artikkel:datatilsynet_etterretningstjenesten}

\subsection{Hva er overvåkning?}
Å finne en konkret definisjon på hva overvåkning er, kan være vanskelig. Det finnes nemlig flere måter å overvåke på, og definisjonen kan variere fra situasjon til situasjon. En generell definisjon kan derfor for eksempel være denne fra civita ``\textit{Overvåkning er å observere, registrere og/eller lagre informasjon om objekter, fenomener, grupper og/eller individer.}'' \parencite{artikkel:civita_overvåkning}, mens en definisjon mer rettet mot digital overvåkning kan være denne ``\textit{Computer and network surveillance is monitoring of computer activity and data stored on a hard drive, or data being transferred over computer networks such as the Internet.}'' \parencite{artikkel:wikipedia_overvåkning}

Denne oppgaven skal ta for seg overvåkning i kontekst av bedrifter eller myndigheter som samler opp eller bruker data om kunder eller innbyggere, og enkeltpersoner som ``spionerer'' på hverandre gjennom digitale platformer som for eksempel tjenesten Snapmap til Snapchat.

\newpage