\section{Teori om temaet}

\subsection{Hva er personvern?}
Å ha en riktig definisjon på personvern er viktig for å ha en korrekt forståelse av hva personvern og personopplysninger er. Så å definere personvern er en oppgave for de som jobber med det til daglig, som for eksempel regjeringen og Datatilsynet. De har henholdsvis definert personvern slik:

\textit{``Personvern er derfor nært knyttet til enkeltindividers muligheter for privatliv, selvbestemmelse og selvutfoldelse.''} \parencite{artikkel:regjeringen_personvern}

\textit{``Personvern handler om retten til et privatliv og retten til å bestemme over egne personopplysninger.''} \parencite{artikkel:datatilsynet_personvern}

Begge disse definisjonene sier omtrent det samme, og legger fram at personvern er nært knyttet til privatliv og selvbestemmelse over egne personopplysninger.

\subsection{Hva er GDPR?}
\textit{The General Data Protection Regulation}, også kjent som GDPR, er i følge EU ``\emph{the toughest privacy and security law in the world}''. Loven trådte i kraft 25. mai 2018, og har som formål å styrke personvernet til innbyggere i EU ved å gjøre brukerne mer opplyst over hva slags data som blir samlet inn, og gjøre det enkelt å laste ned eller slette dataen. Selv om loven ble utarbeidet og trådt i kraft av EU gjelder den alle organisasjoner uavhengig av landstilhørighet, så lenge de behandler eller samler opp data relatert til innbyggere av EU. Dette gjør at loven også gjelder for land som ikke med medlem av EU, som for eksempel Norge, eller land som ikke ligger i Europa som for eksempel USA. \parencite{artikkel:eu_gdpr}

I tillegg ble det i Norge, to måneder etter at GDPR trådte i kraft, vedtatt og trådt i kraft en ny personopplysningslov som gjennomfører personvernforordningen fra EU, altså GDPR, i Norge. Dette vil si at GDPR loven da gjelder i Norge uavhengig av den som er gitt sentralt fra EU. \parencite{artikkel:regjeringen_gdpr}

\subsection{Den nye etterretningstjenesteloven i Norge}

\parencite{artikkel:regjeringen_nylov}

\subsection{Hva er overvåkning?}

\newpage