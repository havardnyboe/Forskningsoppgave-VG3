\section{Resultater og Analyse}

\subsection{Om informantene}
% \subsubsection{Kjønn og alder}
\begin{figure}[H]
    \centering
    \begin{tikzpicture}
        \pie[color = {blue!65!green,red!60,yellow!60,green!60!black,orange!70,teal!50},
            text = inside,
            radius = 2.5
        ]{
            41.6  /Gutt,
            58.4  /Jente}
        \pie[color = {blue!65!green,red!60,yellow!60,green!60!black,orange!70,teal!50},
            explode = {0.3, 0, 0, 0, 0, 0},
            pos = {6,0},
            text = legend,
            radius = 2.5,
            rotate = 25
        ]{
            62.3 /16-19,
            2.6  /20-29,
            7.8  /30-39,
            20.8 /40-49,
            5.2  /50-59,
            1.3  /60+}
    \end{tikzpicture}
    \caption{Fordeling av kjønn og alder blant informantene}
\end{figure}
I alt svarte 77 personer på undersøkelsen, hvor fordelingen av kjønn ble ganske likt.
41,6\% var gutter og 58,4\% var jenter. Som forventet var det størst deltagelse fra aldersgruppen 16-19 da de fleste elevene er i denne gruppen.

% \subsubsection{Fordeling av elever, lærere og trinn}
\begin{figure}[H]
    \centering
    \begin{tikzpicture}
        \pie[color = {blue!65!green,red!60,yellow!60,green!60!black,orange!70,teal!50},
            radius = 2.5,
            text = inside
        ]{
            64.9 /Elev,
            35.1 /Lærer}

        \pie[color = {blue!65!green,red!60,yellow!60,green!60!black,orange!70,teal!50},
            text = legend,
            explode = {0, 0, 0, 0.3},
            pos = {6,0},
            radius = 2.5
        ]{
            18.2  /1. Klasse,
            14.3  /2. Klasse,
            31.2  /3. Klasse,
            36.4  /Andre}
    \end{tikzpicture}
    \caption{Fordeling av elever, lærere og trinn blant informantene}
\end{figure}
Fordelingen av elever og lærere ente opp med å bli omtrent 65\% elever (50 personer) og 35\% lærere (27 personer). I løpet av prosessen fulgte jeg med på deltakelsen og fordelingen av informantene og valgte underveis å markedsføre undersøkelsen direkte til lærere, i tillegg til den generelle markedsføringen, for å få opp deltakelsen blant lærerne.

På de forskjellige trinnene var deltakelsen blant elevene i 3. Klasse nesten dobbelt så stor som de andre trinnene. Uten om at 3. Klasse var mer opplyst om forskningsoppgavene ser jeg ingen grunn enn tilfeldigheter at deltakelsen ikke var mer jevn.

% \subsubsection{Linjefordeling av informantene}
\begin{figure}[H]
    \centering
    \begin{tikzpicture}[every legend entry/.append style={text width=2cm}]
        \pie[color = {blue!65!green,red!60,yellow!60,green!60!black,orange!70,teal!50},
            text = legend,
            explode = {0.3, 0, 0, 0.4, 0, 0}
        ]{
            36.4  /Studiespesialisering,
            3.9   /Kunst design og arkitektur,
            9.1   /Idrett,
            16.9  /Informasjonsteknologi og medieproduksjon,
            1.3   /Teknologi- og industrifag,
            32.5  /Andre eller ikke elev}
    \end{tikzpicture}
    Lærere og elever uten en linje går under kategorien "Andre".
    \caption{Fordeling av linjene blant informantene.}
\end{figure}
Ikke overraskende var det at Studiespesialisering var linjen med flest representanter. Studiespesialisering er den linjen med flest elever på Porsgrunn VGS, og er i tillegg der forskningsoppgavene er mest kjent. 

Litt overraskende var det at informasjonsteknologi og medieproduksjon ble en så klar andreplass. Kanskje er det fordi temaet et mer relevant for de, eller fordi veilederen min oppfordret de til å delta i undersøkelsen.

\subsection{Del 1 - Generelt om personvern}
% \subsubsection{Kjennskap til GDPR}
\begin{figure}[H]
    \centering
    \begin{tikzpicture}
        \pie[color = {blue!65!green,red!60,yellow!60,green!60!black,orange!70,teal!50},
            text = legend
        ]{
            19.5  /Ja har hørt om og kjenner til de,
            41.5  /Har hør om men kjenner ikke til de,
            39    /Nei har ikke hørt om og kjenner ikke til de}
    \end{tikzpicture}
    \caption{Kjenner du til de europeiske lovendringene som skjedde i 2018? (GDPR)}
\end{figure}
Flertallet av informantene kjente til eller hadde hørt om de europeiske lovendringene, også kjent som GDPR. Totalt av de 77 informantene var det 47 som svarte at de kjente til lovendringene. Her var fordelingen av lærere og elever også veldig likt 25 av informantene som kjente til GDPR var lærere og de resterende 22 var elever. 

Når det gjelder informantene som svarte at de ikke hadde hørt om eller kjenner til lovendringene av kun to av disse lærere og resten, altså 28 stykker, elever. En grunn til at nesten alle lærerne kjente til lovendringene, og kun omtrent halvparten av elevene, kan være at når lovendringene trådte i kraft i 2016 og ble omtalt i mediene, var de fleste elevene runt 11 til 14 år gamle og fulgte muligens ikke så mye med på reguleringene av europeiske lover. 

% \subsubsection{Nedlasting, sletting og utnytting av data}
\begin{figure}[H]
    \centering
    \begin{tikzpicture}
        \pie[color = {blue!65!green,red!60,yellow!60,green!60!black,orange!70,teal!50},
            radius = 2.5,
            rotate = 270
        ]{
            24.7   /Ja,
            75.3   /Nei}
        \pie[color = {blue!65!green,red!60,yellow!60,green!60!black,orange!70,teal!50},
            pos = {8, 0},
            radius = 2.5
        ]{
            55.8    /Ja,
            44.2    /Nei}
    \end{tikzpicture}
    \begin{itemize}
        \item Har du benyttet deg av muligheten til å laste ned eller slette personlig data?
        \item Er du redd for at personlig data samlet opp om deg skal brukes for å utnytte deg?
    \end{itemize}
    \caption{Nedlasting, sletting og utnytting av data}
\end{figure}
I spørsmålet om informantene hadde benyttet seg av muligheten til å laste ned eller slette personlig data svarte hele $\frac{3}{4}$ at de ikke hadde benyttet seg av den muligheten. Blant disse 58 som svarte nei var 32 av disse elever og 26 lærere. Fordelingen her er omtrent 55\% elever og 45\% lærere, og fordelingen på informantene som svarte ja er omtrent 95\% elever (18 personer) og 5\% lærere (én person). Ut ifra dette kan vi anta at elevene i mye større grad benytter seg av å laste ned eller slette personlig data enn lærere, noe som styrker hypotesen om at elevene er mer opplyst om innsamling og oppbevaring av personopplysninger, som var lagt fram i kapittel \ref{subsubsec:hypoteser}.

Når informantene ble spurt om de var redde for at personlig data samlet opp om de ble brukt for å utnytte de (i denne konteksten oppsporing, overvåkning o.l), var svarene omtrent likt fordelt med 55.8\% på ja-siden og 44.2\% på nei-siden. Blant de 43 som svarte ja var omtrent 72\% (31 personer) elever og omtrent 28\% (12 personer) lærere. Av de 34 nei-siden var omtrent 56\% elever (19 personer) og 44\% (15 personer) lærere. Ut i fra at flertallet av elevene svarte ja, og flertallet av lærerne svarte nei, kan vi anta at elevene, som vi tidligere fant ut var mer opplyste om oppsamlet personopplysninger, er mer engstelige for at personopplysningene skal utnyttes til overvåkning, oppsporing eller liknende. Dette kan bety at jo mer man vet om hva som samles opp av personopplysninger, jo mer engstelig blir man for at opplysningene blir utnyttet til overvåkning.

% \subsubsection{Kjennskap til ny etterretningstjenestelov}
\begin{figure}[H]
    \centering
    \begin{tikzpicture}
        \pie[color = {blue!65!green,red!60,yellow!60,green!60!black,orange!70,teal!50},
            text = legend
        ]{
            6.5   /Ja har hørt om og kjenner til den,
            23.4  /Har hørt om men kjenner ikke til den,
            70.1  /Nei har ikke hørt om og kjenner ikke tilden}
    \end{tikzpicture}
    \caption{Kjenner du til den nye etterretningstjenesteloven som trådte i kraft i Norge sommeren 2020?}
\end{figure}

% \subsubsection{A}
\begin{figure}[H]
    \centering
    \begin{tikzpicture}
        \pie[color = {blue!65!green,red!60,yellow!60,green!60!black,orange!70,teal!50},
            radius = 2.5,
            rotate = 270
        ]{
            48.1   /Ja,
            51.9   /Nei}
        \pie[color = {blue!65!green,red!60,yellow!60,green!60!black,orange!70,teal!50},
            pos = {8, 0},
            radius = 2.5
        ]{
            24.7    /Ja,
            75.3    /Nei}
    \end{tikzpicture}
    \begin{itemize}
        \item Ser du på deg selv som interessert i personvern?
        \item Er du redd for at norske myndigheter skal bruke data de kan samle inn om deg til å utnytte deg?
    \end{itemize}
    \caption{Personverninteresse og norske myndigheter}
\end{figure}



\subsection{Del 2 - Overvåkning}
% \subsubsection{A}
\begin{figure}[H]
    \centering
    \begin{tikzpicture}
        \pie[color = {blue!65!green,red!60,yellow!60,green!60!black,orange!70,teal!50},
            text = legend,
            rotate = 68
        ]{
            31.2 /Flere ganger daglig (mer enn 2 timer),
            39   /Noen ganger daglig (1-2 timer),
            3.9  /Flere ganger i uken (mer en 2 timer),
            9.1  /Noen ganger i uken (under 2 timer),
            2.6  /Noen ganger i måneden,
            14.3 /Aldri Har ikke Snapchat}
    \end{tikzpicture}
    \caption{Hvor ofte bruker du Snapchat?}
\end{figure}

% \subsubsection{A}
\begin{figure}[H]
    \centering
    \begin{tikzpicture}
        \pie[color = {blue!65!green,red!60,yellow!60,green!60!black,orange!70,teal!50},
            text = legend
        ]{
            20.3   /Er synlig for alle venner,
            52.2   /Er synlig for utvagte venner,
            27.5   /Er ikke synlig for noen venner}
    \end{tikzpicture}
    \caption{Er du synlig for dine venner på Snapmap?}
\end{figure}

\begin{figure}[H]
    \centering
    \begin{tikzpicture}
        \pie[color = {blue!65!green,red!60,yellow!60,green!60!black,orange!70,teal!50},
            text = legend
        ]{
            13    /Hver gang,
            14.5  /Noen ganger,
            55.1  /Skjeldnere,
            17.4  /Aldri}
    \end{tikzpicture}
    \caption{Når du er på Snapchat, hvor ofte sjekker du Snapmap?}
\end{figure}

% \subsubsection{A}
\begin{figure}[H]
    \centering
    \begin{bchart}[max=28, step=4]
        \bcbar[label=Hvor nære venner er,
            color=blue!65!green]{25}
        \smallskip
        \bcbar[label=Når nære venner var pålogget,
            color=red!60]{16}
        \smallskip
        \bcbar[label=Hvor familie er,
            color=yellow!60]{12}
        \smallskip
        \bcbar[label=Når familie var pålogget,
            color=green!60!black]{2}
        \smallskip
        \bcbar[label=Hvor andre venner er,
            color=orange!70]{16}
        \smallskip
        \bcbar[label=Når andre venner var pålogget,
            color=teal!50]{11}
        \smallskip
        \bcbar[label=Bruker ikke Snapmap,
            color=purple!70]{5}
    \end{bchart}
    \caption{Når du sjekker Snapmap, hva sjekker du?}
\end{figure}

% \subsubsection{A}
\begin{figure}[H]
    \centering
    \begin{tikzpicture}
        \pie[color = {blue!65!green,red!60,yellow!60,green!60!black,orange!70,teal!50},
        ]{
            19.1  /Ja,
            80.9  /Nei}
    \end{tikzpicture}
    \caption{Føler du at du overvåker andre på Snapmap?}
\end{figure}

% \subsubsection{A}
\begin{figure}[H]
    \centering
    \begin{tikzpicture}
        \pie[color = {blue!65!green,red!60,yellow!60,green!60!black,orange!70,teal!50},
        ]{
            23.9  /Ja,
            76.1  /Nei}
    \end{tikzpicture}
    \caption{Tror du andre overvåker deg på Snapmap?}
\end{figure}

\newpage