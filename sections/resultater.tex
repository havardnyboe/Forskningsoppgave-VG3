\section{Resultater og Analyse}\label{resultater}

\subsection{Om informantene}
\begin{figure}[H]
    \centering
    \begin{tikzpicture}
        \pie[color = {blue!65!green,red!60,yellow!60,green!60!black,orange!70,teal!50},
            text = inside,
            radius = 2.5
        ]{
            41.6  /Gutt,
            58.4  /Jente}
        \pie[color = {blue!65!green,red!60,yellow!60,green!60!black,orange!70,teal!50},
            explode = {0.3, 0, 0, 0, 0, 0},
            pos = {6,0},
            text = legend,
            radius = 2.5,
            rotate = 25
        ]{
            62.3 /16-19,
            2.6  /20-29,
            7.8  /30-39,
            20.8 /40-49,
            5.2  /50-59,
            1.3  /60+}
    \end{tikzpicture}
    \caption{Fordeling av kjønn og alder blant informantene}
\end{figure}
I alt svarte 77 personer på undersøkelsen, hvor fordelingen av kjønn ble ganske likt.
41,6\% var gutter og 58,4\% var jenter. Som forventet var det størst deltagelse fra aldersgruppen 16-19 da de fleste elevene er i denne gruppen.

\begin{figure}[H]
    \centering
    \begin{tikzpicture}
        \pie[color = {blue!65!green,red!60,yellow!60,green!60!black,orange!70,teal!50},
            radius = 2.5,
            text = inside
        ]{
            64.9 /Elev,
            35.1 /Lærer}

        \pie[color = {blue!65!green,red!60,yellow!60,green!60!black,orange!70,teal!50},
            text = legend,
            explode = {0, 0, 0, 0.3},
            pos = {6,0},
            radius = 2.5
        ]{
            18.2  /1. Klasse,
            14.3  /2. Klasse,
            31.2  /3. Klasse,
            36.4  /Andre}
    \end{tikzpicture}
    \caption{Fordeling av elever, lærere og trinn blant informantene}
\end{figure}
Fordelingen av elever og lærere ente opp med å bli omtrent 65\% elever (50 personer) og 35\% lærere (27 personer). I løpet av prosessen fulgte jeg med på deltakelsen og fordelingen av informantene og valgte underveis å markedsføre undersøkelsen direkte til lærere, i tillegg til den generelle markedsføringen, for å få opp deltakelsen blant lærerne. Blant elevene var 50\% gutter og 50\% jenter, og blant lærere var 74\% jenter og 26\% gutter. 

På de forskjellige trinnene var deltakelsen blant elevene i 3. Klasse nesten dobbelt så stor som de andre trinnene. Uten om at 3. Klasse var mer opplyst om forskningsoppgavene ser jeg ingen grunn enn tilfeldigheter at deltakelsen ikke var mer jevn.

\begin{figure}[H]
    \centering
    \begin{tikzpicture}[every legend entry/.append style={text width=2cm}]
        \pie[color = {blue!65!green,red!60,yellow!60,green!60!black,orange!70,teal!50},
            text = legend,
            explode = {0.3, 0, 0, 0.4, 0, 0}
        ]{
            36.4  /Studiespesialisering,
            3.9   /Kunst design og arkitektur,
            9.1   /Idrett,
            16.9  /Informasjonsteknologi og medieproduksjon,
            1.3   /Teknologi- og industrifag,
            32.5  /Andre eller ikke elev}
    \end{tikzpicture}
    Lærere og elever uten en linje går under kategorien "Andre".
    \caption{Fordeling av linjene blant informantene.}
\end{figure}
Ikke overraskende var det at Studiespesialisering var linjen med flest representanter. Studiespesialisering er den linjen med flest elever på Porsgrunn VGS, og er i tillegg der forskningsoppgavene er mest kjent. 

Litt overraskende var det at informasjonsteknologi og medieproduksjon ble en så klar andreplass. Kanskje er det fordi temaet et mer relevant for de, fordi det ligger nært mye av fagstoffet de jobber med.

\subsection{Del 1 - Generelt om personvern}
\begin{figure}[H]
    \centering
    \begin{tikzpicture}
        \pie[color = {blue!65!green,red!60,yellow!60,green!60!black,orange!70,teal!50},
            text = legend
        ]{
            19.5  /Ja har hørt om og kjenner til de,
            41.5  /Har hør om men kjenner ikke til de,
            39    /Nei har ikke hørt om og kjenner ikke til de}
    \end{tikzpicture}
    \caption{Kjenner du til de europeiske lovendringene som skjedde i 2018? (GDPR)}
\end{figure}
Flertallet av informantene kjente til eller hadde hørt om de europeiske lovendringene, også kjent som GDPR. Totalt av de 77 informantene var det 47 som svarte at de kjente til lovendringene. Her var fordelingen av lærere og elever også veldig lik, 25 av informantene som kjente til GDPR var lærere og de resterende 22 var elever. 

Når det gjelder informantene som svarte at de ikke hadde hørt om eller kjenner til lovendringene av kun to av disse lærere og resten, altså 28 stykker, elever. En grunn til at nesten alle lærerne kjente til lovendringene, og kun omtrent halvparten av elevene, kan være at når lovendringene trådte i kraft i 2018 og ble omtalt i mediene, var de fleste elevene rundt 13 til 15 år gamle og fulgte muligens ikke så mye med på reguleringene av europeiske lover. 

\begin{figure}[H]
    \centering
    \begin{tikzpicture}
        \pie[color = {blue!65!green,red!60,yellow!60,green!60!black,orange!70,teal!50},
            radius = 2.5,
            rotate = 270
        ]{
            24.7   /Ja,
            75.3   /Nei}
        \pie[color = {blue!65!green,red!60,yellow!60,green!60!black,orange!70,teal!50},
            pos = {8, 0},
            radius = 2.5
        ]{
            55.8    /Ja,
            44.2    /Nei}
    \end{tikzpicture}
    \begin{itemize}
        \item Har du benyttet deg av muligheten til å laste ned eller slette personlig data?
        \item Er du redd for at personlig data samlet opp om deg skal brukes for å utnytte deg?
    \end{itemize}
    \caption{Nedlasting, sletting og utnytting av data}
\end{figure}
I spørsmålet om informantene hadde benyttet seg av muligheten til å laste ned eller slette personlig data svarte hele $\frac{3}{4}$ at de ikke hadde benyttet seg av den muligheten. Blant disse 58 som svarte nei var 32 av disse elever og 26 lærere. Fordelingen her er omtrent 55\% elever og 45\% lærere, og fordelingen på informantene som svarte ja er omtrent 95\% elever (18 personer) og 5\% lærere (én person). Ut ifra dette kan vi anta at elevene i mye større grad benytter seg av å laste ned eller slette personlig data enn lærere, noe som styrker hypotesen om at elevene er mer opplyst om innsamling og oppbevaring av personopplysninger, som var lagt fram i kapittel \ref{subsubsec:hypoteser}.

Når informantene ble spurt om de var redde for at personlig data samlet opp om de ble brukt for å utnytte de (i denne konteksten oppsporing, overvåkning o.l), var svarene omtrent likt fordelt med 55,8\% på ja-siden og 44,2\% på nei-siden. Blant de 43 som svarte ja var omtrent 72\% (31 personer) elever og omtrent 28\% (12 personer) lærere. Av de 34 på nei-siden var omtrent 56\% elever (19 personer) og 44\% (15 personer) lærere. Ut i fra at flertallet av elevene svarte ja, og flertallet av lærerne svarte nei, kan vi anta at elevene, som vi tidligere fant ut var mer opplyste om oppsamlet personopplysninger, er mer engstelige for at personopplysningene skal utnyttes til overvåkning, oppsporing eller liknende. Dette kan bety at jo mer man vet om hva som samles opp av personopplysninger, jo mer engstelig blir man for at opplysningene blir utnyttet til overvåkning.

\begin{figure}[H]
    \centering
    \begin{tikzpicture}
        \pie[color = {blue!65!green,red!60,yellow!60,green!60!black,orange!70,teal!50},
            text = legend,
            explode = {0, 0, 0.4},
            radius = 2.5
        ]{
            6.5   /Ja har hørt om og kjenner til den,
            23.4  /Har hørt om men kjenner ikke til den,
            70.1  /Nei har ikke hørt om og kjenner ikke til den}
    \end{tikzpicture}
    \caption{Kjenner du til den nye etterretningstjenesteloven som trådte i kraft i Norge sommeren 2020?}
    \label{fig:kjenner_loven}
\end{figure}
Når informantene ble spurt om de kjente til den nye etterretningstjenesteloven som trådte i kraft i Norge sommeren 2020, svarte 70,1\% (54 personer) at de ikke hadde hørt om eller kjente til den. Av de 54 som svarte nei var 76\% (41 personer) elever og 24\% (13 personer lærere). Av de 23 som hadde hørt om eller kjente til loven var 39\% (9 personer) elever og 61\% (14 personer) lærere. Ut i fra dette kan vi klart se at lærerne er mer informert om lovendringen enn elevene. Grunnen til dette er  vanskelig å fastslå, men en grunn kan være at lærere, som tilhører en eldre aldersgruppe, får med seg generelt mer  varierte nyheter enn elevene som oppdaterer seg mest på de nyhetene som engasjerer de mest.

\begin{figure}[H]
    \centering
    \begin{tikzpicture}
        \pie[color = {blue!65!green,red!60,yellow!60,green!60!black,orange!70,teal!50},
            radius = 2.5,
            rotate = 270
        ]{
            48.1   /Ja,
            51.9   /Nei}
        \pie[color = {blue!65!green,red!60,yellow!60,green!60!black,orange!70,teal!50},
            pos = {8, 0},
            radius = 2.5
        ]{
            24.7    /Ja,
            75.3    /Nei}
    \end{tikzpicture}
    \begin{itemize}
        \item Ser du på deg selv som interessert i personvern?
        \item Er du redd for at norske myndigheter skal bruke data de kan samle inn om deg til å utnytte deg?
    \end{itemize}
    \caption{Personverninteresse og norske myndigheter}
\end{figure}
Når informantene ble spurt om de så på seg selv som interessert i personvern var svarene delt nesten helt på midten. Av de 37 som svarte ja var var 54\% (20 personer) elever og 46\% (17 personer) lærere. Av de 40 som svarte nei var 75\% (30 personer) elever og 25\% (10 personer) lærere. Her kan det også være interessant å se på fordelingen av kjønn på de ulike svarene. Av de 37 som svarte ja var 38\% (14 personer) gutter og 62\% (23 personer) jenter. Av de 40 som svarte nei var 45\% (18 personer) gutter og 55\% (22 personer) jenter. Her ser vi at av de som svarte ja var det ingen stor forskjell på elver og lærere, men det var et større antall jenter. Av de som svarte nei var det en mye større andel elver, mens kjønnsfordelingen var ganske lik. 

I spørsmålet om informantene var redde for at norske myndigheter skulle bruke data de kan samle inn om informantene til å utnytte de svarte 75\% at de ikke var redde for dette. Av de 58 som svare nei var 62\% (36 personer) elever og 38\% (22 personer) lærere. Av de 19 som svare ja var 74\% (14 personer) elever og 26\% (5 personer) lærere. Her er det mest interessant så sammenlikne svarene med resultatene fra Figur \ref{fig:kjenner_loven}. Av de 58 som svare at de ikke var redde for at innsamlet data skulle misbrukes svarte 34,5\% (20 personer) at de hadde hørt om eller kjente til loven, og 65,5\% (38 personer) at de verken hadde hørt om eller kjente til loven. Av de 19 som svarte at de var redd for at innsamlet data skulle misbrukes svarte 84\% (16 personer) at de verken kjente til eller hadde hørt om loven, mens 16\% (3 personer) svarte at de hadde hørt om eller kjente til loven. Her ser vi tydelig at de som kjenner til eller har hørt om loven er mindre redde for at dataen skal misbrukes, og de som ikke kjenner til den er mer redde.

\subsection{Del 2 - Overvåkning}
\begin{figure}[H]
    \centering
    \begin{tikzpicture}
        \pie[color = {blue!65!green,red!60,yellow!60,green!60!black,orange!70,teal!50},
            text = legend,
            rotate = 68
        ]{
            31.2 /Flere ganger daglig (mer enn 2 timer),
            39   /Noen ganger daglig (1-2 timer),
            3.9  /Flere ganger i uken (mer en 2 timer),
            9.1  /Noen ganger i uken (under 2 timer),
            2.6  /Noen ganger i måneden,
            14.3 /Aldri Har ikke Snapchat}
    \end{tikzpicture}
    \caption{Hvor ofte bruker du Snapchat?}
\end{figure}
Når informantene ble spurt hvor ofte de brukte Snapchat svarte 70\% (54 personer) at de brukte Snapchat daglig, 13\% (10 personer) at de brukte Snapchat ukentlig, og 17\% (13 personer) at de månedlig eller aldri brukte Snapchat. Av de som brukte Snapchat mest var 80\% (43 personer) elever og 20\% (11 personer) lærere. Av de som brukte Snapchat ukentlig var 60\% (6 personer) elever og 40\% (4 personer) lærere. Av de som brukte Snapchat minst var 8\% (1 person) elever og 92\% (12 personer) lærere. Ut i fra dette ser vi klart at elvene bruker Snapchat i mye større grad enn lærerne. Dette gjør at det blir vanskelig å sammenlikne lærerne og elevene videre i analysen siden 77\% av de som bruker Snapchat er elever og 23\% er lærere.

\begin{figure}[H]
    \centering
    \begin{tikzpicture}
        \pie[color = {blue!65!green,red!60,yellow!60,green!60!black,orange!70,teal!50},
            text = legend
        ]{
            20.3   /Er synlig for alle venner,
            52.2   /Er synlig for utvagte venner,
            27.5   /Er ikke synlig for noen venner}
    \end{tikzpicture}
    \caption{Er du synlig for dine venner på Snapmap?}
\end{figure}
I Snapmap tjenesten på Snapchat har man muligheten til å velge hvem av sine venner man er synlig for på kartet. Når informantene ble spurt om de var synlig for venne sine på Snapmap svarte 20,3\% (14 personer) at de var synlig for alle venner, 52,2\% (36 personer) at de var synlig for utvalgte venner og 27,5\% (19 personer) at de ikke var synlig for noen venner.

\begin{figure}[H]
    \centering
    \begin{tikzpicture}
        \pie[color = {blue!65!green,red!60,yellow!60,green!60!black,orange!70,teal!50},
            text = legend
        ]{
            13    /Omtrent hver gang,
            14.5  /Noen ganger,
            55.1  /Sjeldnere,
            17.4  /Aldri}
    \end{tikzpicture}
    \caption{Når du er på Snapchat, hvor ofte sjekker du Snapmap?}
\end{figure}
I spørsmålet om hvor ofte informantene sjekket Snapmap når de var på Snapchat svarte 13\% (9 personer) at de sjekket Snapmap omtrent hver gang, 14,5\% (10 personer) sjekket noen ganger, 55,1\% (38 personer) sjekket sjeldnere og 17,4\% (12 personer) sjekket aldri Snapmap.

Av de som sjekket oftest, omtrent hver gang eller noen ganger, var 26\% (5 personer) gutter og 74\% (14 personer) jenter. Av de som sjekket sjeldnere var 50\% (19 personer) gutter og 50\% (19 personer) jenter. Av de som aldri sjekket Snapmap var 25\% (3 personer) gutter og 75\% (9 personer) jenter. 

\begin{figure}[H]
    \centering
    \begin{bchart}[max=28, step=4]
        \bcbar[label=Hvor nære venner er,
            color=blue!65!green]{25}
        \smallskip
        \bcbar[label=Når nære venner var pålogget,
            color=red!60]{16}
        \smallskip
        \bcbar[label=Hvor familie er,
            color=yellow!60]{12}
        \smallskip
        \bcbar[label=Når familie var pålogget,
            color=green!60!black]{2}
        \smallskip
        \bcbar[label=Hvor andre venner er,
            color=orange!70]{16}
        \smallskip
        \bcbar[label=Når andre venner var pålogget,
            color=teal!50]{11}
        \smallskip
        \bcbar[label=Bruker ikke Snapmap,
            color=purple!70]{5}
    \end{bchart}
    \caption{Når du sjekker Snapmap, hva sjekker du?}
\end{figure}
Når informantene ble spurt om hva de sjekket når de var på Snapmap, sjekket de fleste hvor nære venner var når når de sist var pålogget. Etter det var det å sjekke hvor andre venner var, hvor noen i familien var, og når andre var pålogget som var mest vanlig. 

\begin{figure}[H]
    \centering
    \begin{tikzpicture}
        \pie[color = {blue!65!green,red!60,yellow!60,green!60!black,orange!70,teal!50},
        ]{
            19.1  /Ja,
            80.9  /Nei}
    \end{tikzpicture}
    \caption{Føler du at du overvåker andre på Snapmap?}
\end{figure}
Når informantene ble spurt om de følte at de overvåket andre på Snapmap svarte 81\% at de ikke følte at de overvåket andre, mens 19\% svarte at de følte det. Av de 55 som svarte nei var 44\% (24 personer) gutter og 56\% (31 personer) jenter. Av de 13 som svarte ja var 23\% (3 personer) gutter og 77\% (10 personer) jenter. Dette vister at det generelt er det få fra begge kjønnene som føler de overvåker andre på Snapmap, men av de som føler at de overvåker andre er det en større andel jenter enn gutter.

\begin{figure}[H]
    \centering
    \begin{tikzpicture}
        \pie[color = {blue!65!green,red!60,yellow!60,green!60!black,orange!70,teal!50},
        ]{
            23.9  /Ja,
            76.1  /Nei}
    \end{tikzpicture}
    \caption{Tror du andre overvåker deg på Snapmap?}
\end{figure}
I spørsmålet om informantene trodde andre overvåket dem på Snapmap svarte 76\% at de ikke trodde andre overvåket dem, mens 24\% svarte at de trodde det. Hvis vi sammenlikner dette med resultatene fra forrige figur viser det at flere av informantene føler mere på at andre overvåker dem, enn at de overvåker andre.

Av de 51 som ikke trodde at andre overvåket dem var 43\% (22 personer) gutter og 57\% (29 personer) jenter. Av de 16 som trodde andre overvåket dem på Snapmap var 31,25\% (5 personer) gutter og 68,75\% (11 personer) jenter.

\newpage