\section{Metode}

\subsection{Kvantitativ metode}
Det er valgt kvantitativ metode for oppgaven for å kunne få inn et stort omfang av data som kan analyseres og undersøkes. Den kvantitative metoden vil derfor fungere bedre til å svare på siste del av problemstillingen i kapittel \ref{subsubsec:problemstilling}.

Det skal også brukes en den tekster i oppgaven for å prøve å svare på den første delen av av problemstillingen i kapittel \ref{subsubsec:problemstilling}. Tekstene som skal brukes er blant annet bøker, nettartikler og tidligere forskning.

\subsection{Utstyrsliste}
Under er en liste med utstyr som ble brukt i forbindelse med forskningsoppgaven.
\begin{itemize}
    \item Google Forms
    \item pgfplots
    \item pgf-pie
\end{itemize}
Undersøkelsen ble laget i Google Forms og resultatene ble samlet opp i et regneark.

\subsection{Informantene}
Informantene er elever og ansatte ved Porsgrunn VGS i alderen 16 og opp. Aldersgruppene er såpass bred for å kunne ha mulighet til å sammenlikne de ulike aldersgruppene, og sette deres svar opp mot hverandre. Aldersgruppene som ble valgt for undersøkelsen er

\begin{table}[h]
    \begin{center}
        \begin{tabular}{|c|c|c|c|c|c|}
            \hline
            Gruppe 1 & Gruppe 2 & Gruppe 3 & Gruppe 4 & Gruppe 5 & Gruppe 6 \\
            \hline
            16-19 & 20-29 & 30-39 & 40-49 & 50-59 & 60+ \\
            \hline
        \end{tabular}
        \caption{Aldersgruppene i spørreundersøkelsen}
    \end{center}
\end{table}

Det er valgt et kortere intervall i Gruppe 1 både for å samle alle elevene i en gruppe, og fordi antall elever er mye større enn antall ansatte. Så hvis det blir noen grupper der det er for få deltakere til å kunne sammenlikne, kan gruppene 2-6 kombineres til en \textit{Ansatte} gruppe. Aldersgruppene er valgt i henhold til retningslinjene til \textit{Norsk senter for forskningsdata} \parencite{artikkel:nsd_forsknig}. Informantene deltar via en frivillig spørreundersøkelse over nett, der alle spørsmålene som omhandler temaet er frivillig å svare på. 

\subsection{Forskningsteknikk og feilkilder}
\subsubsection{Forskningsteknikk}
\begin{itemize} 
    \item Spørreundersøkelse (kvantitativ)
    \item Tekst (Nettartikler og Bøker)
    \item Nedlasting av egne personopplysninger
    \item Statistikk
\end{itemize}

\subsubsection{Feilkilder}
\begin{itemize}
    \item Flere elever enn ansatte, vanskelig å sammenlikne
    \item 
\end{itemize}

\subsection{Hvordan og hvorfor går du fram slik?}

\begin{lstlisting}[language=Python, caption=Python eksempel]
    from matplotlib import pyplot as plt
    import numpy as np

    for i in list:
        print(i, "Test")
    
    # Funksjon
    def f(x):
        return np.sqrt(x-1)
\end{lstlisting}

\newpage