\section{Metode}

\subsection{Kvantitativ metode}
Det er valgt kvantitativ metode for oppgaven for å kunne få inn et stort omfang av data som kan analyseres og undersøkes. Den kvantitative metoden vil derfor fungere bedre til å svare på det andre delspørsmålet i kapittel \ref{subsubsec:problemstilling}.

Det skal også brukes en del tekster og artikler i oppgaven for å prøve å svare på det første delspørsmålet i kapittel \ref{subsubsec:problemstilling}. Tekstene som skal brukes er blant annet bøker, nettartikler og tidligere forskning.

\subsection{Utstyrsliste}
Undersøkelsen ble laget i Google Forms, plakaten til undersøkelsen ble laget i Microsoft Word og resultatene ble samlet opp i et Google regneark, alle dokumentene ligger vedlagt i \hyperref[vedlegg]{Vedlegg} kapittelet. Oppgaven ble skrevet i \LaTeX{} med VSCode og grafikken og diagrammene ble laget med pakkene tikzpicture, pgf-pie, og bchart.  

\subsection{Informantene}
Informantene er elever og ansatte ved Porsgrunn VGS i alderen 16 og opp. Aldersgruppene er såpass brede for å kunne ha mulighet til å sammenlikne de ulike aldersgruppene, og sette deres svar opp mot hverandre. Aldersgruppene som ble valgt for undersøkelsen er

\begin{table}[h]
    \begin{center}
        \begin{tabular}{|c|c|c|c|c|c|}
            \hline
            Gruppe 1 & Gruppe 2 & Gruppe 3 & Gruppe 4 & Gruppe 5 & Gruppe 6 \\
            \hline
            16-19 & 20-29 & 30-39 & 40-49 & 50-59 & 60+ \\
            \hline
        \end{tabular}
        \caption{Aldersgruppene i spørreundersøkelsen}
    \end{center}
\end{table}

Det er valgt et kortere intervall i Gruppe 1 både for å samle alle elevene i en gruppe, og fordi antall elever er mye større enn antall ansatte. Så hvis det blir noen grupper der det er for få deltakere til å kunne sammenlikne, kan gruppene 2-6 kombineres til en \textit{Ansatte} gruppe. Aldersgruppene er valgt i henhold til retningslinjene til \textit{Norsk senter for forskningsdata} \parencite{artikkel:nsd_forsknig}. Informantene deltar via en frivillig spørreundersøkelse over nett, der alle spørsmålene som omhandler temaet er frivillig å svare på. 
\vfill

\subsection{Forskningsteknikk}
Hovedmetoden for oppgaven var spørreundersøkelsen som ble gjennomført. Den ble valgt for å få svar på store deler av problemstillingen siden den hovedsakelig omhandlet elevene og lærerne ved Porsgrunn videregående skole. I tillegg ble det brukt en del tekster, hvor flesteparten var nettartikler men noe bøker og annen forskning, for å svare på resten av problemstillingen, og får å kunne gi et grunnleggende fundament til teorien som trengtes for oppgaven.

Det var planlagt å laste ned og analysere reell data som var samlet opp av ulike bedrifter, men underveis ble det konkludert med at dette ville ta for lang tid å analysere og ville bli en for stor jobb til at det var vært det.

\subsection{Feilkilder}
Det er flere feilkilder i oppgaven som kan føre til at resultatene ikke ble helt optimale eller korrekte. Den største feilkilden er nok antallet informanter, eller lettere sagt mangelen på informanter. Til å være en skole med samlet over 1000 elever og ansatte er 77 informanter et for lite tall til å garantert være representativt for hele skolen.

En annen feilkilde som ikke er lett å kontrollere er at man ikke kan vite om informantene har svar ærlig på alle spørsmålene. Det at undersøkelsen er anonym skal i teorien hjelpe med å få mer ærlige svar, men man kan aldri vite helt sikkert.

En annen feilkilde som henger sammen med undersøkelsens anonymitet er at informantene i teorien har hatt muligheten til å svare på undersøkelsen flere ganger, og på den måten påvirke resultatene. For at informantene skal få være helt anonyme er dette et utfall som ikke kan gjøres noe med, så i analysen av resultatene er det antatt at alle svarene er unike og fra ulike informanter.

\newpage