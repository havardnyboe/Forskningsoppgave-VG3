\section{Innledning}

\subsection{Tema og problemstilling}\label{subsec:problemstilling}

\subsubsection{Tema}
Tema for oppgaven er personvern og sikkerhet på nett. Hensikten med oppgaven er å få en oversikt over personopplysninger som blir lagret av oss, og finne ut hva de brukes til. Oppgaven skal også gi svar på hva elever og lærere ved Porsgrunn VGS vet om personvern, hvilke av deres personopplysninger som blir lagret, og hva de brukes til.

\subsubsection{Problemstilling og delspørsmål}\label{subsubsec:problemstilling}
Iløpet av startfasen til denne oppgaven har problemstillingen endret seg litt underveis. Den startet først som en litt vag problemstilling som tok for seg store tema som ble vanskelig å svare på i en kort forskningsoppgave. Den første problemstillingen så slik ut:

\textit{Hvorfor blir det lagret personopplysninger av oss, og vet vi ha de brukes til?}

Etter hvert som metodene og hensikten med oppgaven ble klarere ble problemstillingen avgrenset og innsnevret til å dekke mer konkrete spørsmål som var enklere å undersøke direkte. Den endelige problemstillingen ble derfor:

\textit{Hvor mye bryr vi oss om personvern, og er vi redd for å bli overvåket?}

``Vi'' i denne sammenhengen er målgruppen for undersøkelsen som skal gjennomføres, altså elever og lærere ved Porsgrunn VGS. 

For å lettere kunne svare på problemstillingen er den delt opp i flere delspørsmål som skap undersøkes og svares på underveis. Delspørsmålene er som følger:

\begin{itemize}
    \item Hva er personvern, personopplysninger og overvåkning?
    \begin{itemize}
        \item Hva er personvern?
        \item Hva handler General Data Protection Regulation (GDPR) loven om?
        \item Hva går den nye etterretningstjenesteloven i Norge ut på?
        \item Hva er overvåkning?
    \end{itemize}
    \item Hva vet elever og lærere ved Porsgrunn VGS om personvern og personopplysninger?
    \begin{itemize}
        \item Hva finnes det av lovverk rundt personvern?
        \item Har man tilgang til egne personopplysninger?
        \item Kan enkeltpersoner overvåke hverandre, eller er det bare store bedrifter? 
        \item Hvem overvåker hvem?
    \end{itemize}
\end{itemize}

\subsubsection{Hypoteser}\label{subsubsec:hypoteser}
I forbindelse med spørreundersøkelsen som skal gjennomføres har jeg kommet opp med noen hypoteser om resultatene av undersøkelsen. Disse hypotesene er basert på personlige antakelser og fordommer om temaet og informantene.

\begin{itemize}
    \item Elevene vil være mer opplyst om innsamling og oppbevaring av personopplysninger enn lærerne.
    \item Elever og lærere er mer redd for å bli overvåket av andre tjenester enn av norske myndigheter.
\end{itemize}

\subsection{Tidligere forskning}
Debatten om personvern med tanke på overvåkning, sporing og lagring av informasjon startet allerede i USA på 1960 og 1970-tallet \parencite[36]{bok:nissenbaum}. Det har tidligere blitt forsket på hvordan unge forholder seg til personvern på nett, blant annet av Terje Sandkjær Hanssen i hans masteravhandling. Denne oppgaven skal ta for seg noe av det Hanssen la frem som veien videre i sin masteravhandling. \parencite{master:hanssen}

\subsection{Hva det skal forskes på}
I denne oppgaven skal det forskes på elever og lærere ved Porsgrunn videregående skole sine holdninger til personvern og overvåkning. Det skal gjennomføres en undersøkelse blant elevene og lærerne som skal undersøke hva de vet om personvern, og personopplysningene de legger fra seg på nett og overvåkning basert på hypotesene i kapittel \ref{subsubsec:hypoteser}.

\subsection{Bakgrunnskunnskap}
Siden denne oppgaven skal utforske Hanssens ide om \textit{veien videre} er det viktig å vite hva han la fram som veien videre i forskningen. Under er et utdrag fra kapittel 8.1 i hans masteravhandling:

\textit{``Snapmap er et mikrokosmos av personvernvurderinger, (...) Det hadde vært interessant å utforske brukernes selvbevissthet til tjenesten nærmere, hvordan brukerne «leker» overvåkere av vennene sine, (...) Det kan også være interessant å utforske personvernholdninger på nett ut fra andre alderssegment for å sammenligne hvordan meninger og holdninger avvike. (...) eldre generasjoner [har] andre forutsetninger for å forstå internett og teknologi, og med det også andre holdninger og forventinger om risikomomenter.''} \parencite[69]{master:hanssen}

\newpage