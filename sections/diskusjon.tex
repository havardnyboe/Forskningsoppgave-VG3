\section{Diskusjon}

\subsection{Resultat av hypoteser}
I kapittel \ref{subsubsec:hypoteser} la jeg fram noe hypoteser om resultatene til undersøkelsen basert på egne antakelser og fordommer om temaet og informantene. Etter å ha presentert resultatene og analysen i kapittel \ref{resultater} er det nå mulig å se om hypotesene stemmer.

\subsubsection{Første hypotese}
Den første hypotesen jeg la fram var:

``\textit{Elevene vil være mer opplyst om innsamling og oppbevaring av personopplysninger enn lærerne.}''

Denne hypotesen var basert på min antakelse av at elevene, som tilhører en yngre generasjon, har vokst opp i en verden der internett alltid har eksistert, og det har derfor vært en mer naturlig del av livet deres. Jeg antok derfor at elevene ville være mer opplyst om dette fordi det var en mer naturlig del av livet deres enn lærerne. 

De spørsmålene som er mest knyttet til denne hypotesen er spørsmålene: 

``\textit{Kjenner du til de europeiske lovendringene som skjedde i 2018?}'',\\ ``\textit{Kjenner  du  til  den  nye  etterretningstjenesteloven  som  trådte  i  kraft  i  Norge sommeren 2020?}'' og\\ ``\textit{Ser du på deg selv som interessert i personvern?}''

I alle disse spørsmålene ble svarene veldig like, og i der spørsmålet som skilte seg mest ut (spørsmål nummer to), svarte faktisk flere lærere at de kjente til den nye loven som ble innført. 

Basert på dette kan jeg konkludere med at hypotesen min ikke stemte og at det, på grunnlag av resultatene i denne undersøkelsen, ikke er noe som indikerer at elevene er mer opplyst om innsamling og oppbevaring av personopplysninger enn lærere.

\subsubsection{Andre hypotese}
Den andre hypotesen jeg la fram var:

``\textit{Elever og lærere er mer redd for å bli overvåket av andre tjenester enn av norske myndigheter.}''

Denne hypotesen var basert på min antakelse av at elever og lærere, som nordmenn flest, hadde stor tillit til norske myndigheter og stolte på at de ikke misbrukte innsamlet data. \parencite{video:eia_tillit} Jeg antok derfor at de stolte mindre på andre aktører, og derfor var mer redd for at disse skulle misbruke innsamlet data.
\newpage

De spørsmålene som er mest knyttet til denne hypotesen er spørsmålene:

``\textit{Er du redd for at personlig data samlet opp om deg skal brukes for å utnytte deg?}'' og ``\textit{Er du redd for at norske myndigheter skal bruke data de kan samle inn om deg til å utnytte deg?}''

Svarene på disse spørsmålene ble som antatt på forhånd, og det var flere som svarte at de var redde for at andre aktører skulle misbruke innsamlet data, enn norske myndigheter.

Jeg kan derfor på grunnlag av resultatene i undersøkelsen konkludere med at elevene og lærere er mindre redde for at data samlet inn av norske myndigheter skal brukes til å utnytte dem, enn data samlet inn av andre aktører.

\subsection{Svar på problemstilling}
Tilslutt vil jeg prøve å gi et svar på problemstillingen som jeg la fram i kapittel \ref{subsubsec:problemstilling}. Problemstillingen var som følger: 

``\textit{Hvor mye bryr vi oss om personvern, og er vi redd for å bli overvåket?}''

Jeg skal svare på problemstillingen i to deler, der jeg tar for meg henholdsvis første og siste del av problemstillingen.

\subsubsection{Første del av problemstillingen}
I første del av problemstillingen spør jeg: ``\textit{Hvor mye bryr vi oss om personvern?}''. Som tidligere nevnt er ``Vi'' i denne sammenhengen elever og lærere ved Porsgrunn videregående skole. 

Basert på resultatene til undersøkelsen og svarene på hypotesene kan vi si at elevene og lærerne ikke bryr seg noe utmerket om personvern. Omtrent 40\% hadde ikke hørt om GDPR, 75\% hadde ikke benyttet seg av muligheten til å laste ned eller slette personopplysninger, 70\% hadde ikke hørt om den nye etterretningstjenesteloven i Norge og omtrent 50\% svarte at de ikke så på seg selv som interessert i personvern.

\subsubsection{Siste del av problemstillingen}
I siste del av problemstillingen spør jeg: ``\textit{Er vi redd for å bli overvåket?}''. Det var gitt flere konkrete spørsmål i undersøkelsen som var ment for å gi svar på nettopp denne delen. De spørsmålene var:

``\textit{Er du redd for at personlig data samlet opp om deg skal brukes for å utnytte deg?}'',\\
``\textit{Er du redd for at norske myndigheter skal bruke data de kan samle inn om deg til å utnytte deg?}'',\\
``\textit{Er du synlig for dine venner på Snapmap?}'' og \\
``\textit{Føler du at du overvåker andre på Snapmap?}''
\newpage

Ut i fra svarene informantene ga på de forskjellige spørsmålene kan vi konkludere med at informantene er mer redde for å bli overvåket, jo mindre tillit de har til de som håndterer informasjonen om de. I spørsmålet om de følte at andre overvåket dem på Snapmap svarte et klart flertall at de ikke var det, og grunnen til dette er nok at de fleste kjenner vennene de har på Snapchat. En informant skrev dette som et svar på et av spørsmålene i undersøkelsen, og jeg mener det gir en god begrunnelse på konklusjonen: 

``\textit{Alle vennene jeg har på snap er folk jeg kjenner veldig godt, så snapmap ville ikke vært metoden noen ville brukt for å finne hverandre.}'' - Elev, 16-19 år



\newpage