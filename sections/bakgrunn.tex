\section{Bakgrunn}

\subsection{Tidligere forskning}
Debatten om personvern med tanke på overvåkning, sporing og lagring av informasjon starttet allerede i USA på 1960 og 1970-tallet \parencite[36]{bok:nissenbaum}. Det har tidligere blitt forsket på hvordan unge forholder seg til personvern på nett, blant annet av Terje Sandkjær Hanssen i hans masteravhandling \parencite{master:hanssen}. Denne oppgaven skal utforske det Hanssen la frem som veien videre i sin masteravhandling. \parencite[69]{master:hanssen}

https://www.jus.uio.no/ifp/forskning/om/publikasjoner/complex/2006-2011/complex-2009-04.pdf


\subsection{Hva det skal forskes på}
I denne oppgaven skal det forskes på hvorfor det blir lager personopplysninger av oss, gjennom å prøve å finne svar på problemstillingen og delspørsmålene i kapittel \ref{sec:problemstilling}. Det skal også gjennomføres en undersøkelse blant elever og lærere ved Porsgrunn videregående skole som skal undersøke hva elever og lærere vet om personvern, og personopplysningene de legger fra seg på nett.

\subsection{Bakgrunnskunnskap}
\subsubsection{Hva er personvern?}
Å ha en riktig definisjon på personvern er viktig for å ha en korrekt forståelse av hva personvern og personopplysninger er. Så å definere personvern er en oppgave for de som jobber med det til daglig, som for eksempel regjeringen og Datatilsynet. De har henholdsvis definert personvern slik:

\textit{``Personvern er derfor nært knyttet til enkeltindividers muligheter for privatliv, selvbestemmelse og selvutfoldelse.''} \parencite{regjeringen_personvern}

\textit{``Personvern handler om retten til et privatliv og retten til å bestemme over egne personopplysninger.''} \parencite{datatilsynet_personvern}

\newpage