\section{Bakgrunn}

\subsection{Tidligere forskning}
Debatten om personvern med tanke på overvåkning, sporing og lagring av informasjon startet allerede i USA på 1960 og 1970-tallet \parencite[36]{bok:nissenbaum}. Det har tidligere blitt forsket på hvordan unge forholder seg til personvern på nett, blant annet av Terje Sandkjær Hanssen i hans master-avhandling. Denne oppgaven skal ta for seg noe av det Hanssen la frem som veien videre i sin master-avhandling. \parencite{master:hanssen}

\subsection{Hva det skal forskes på}
I denne oppgaven skal det forskes på hvorfor det blir lagret personopplysninger av oss, gjennom å prøve å finne svar på problemstillingen og delspørsmålene i kapittel \ref{subsec:problemstilling}. Det skal også gjennomføres en undersøkelse blant elever og lærere ved Porsgrunn videregående skole som skal undersøke hva elever og lærere vet om personvern, og personopplysningene de legger fra seg på nett, basert på hypotesene i kapittel \ref{subsec:hypoteser}.

\subsection{Bakgrunnskunnskap}
Siden denne oppgaven skal utforske Hanssens ide om \textit{veien videre} er det viktig å vite hva han la fram som veien videre i forskningen. Under er et utdrag fra kapittel 8.1 i hans master-avhandling:

\textit{``Snapmap er et mikrokosmos av personvernvurderinger, (...) Det hadde vært interessant å utforske brukernes selvbevissthet til tjenesten nærmere, hvordan brukerne «leker» overvåkere av vennene sine, (...) Det kan også være interessant å utforske personvernholdninger på nett ut fra andre alderssegment for å sammenligne hvordan meninger og holdninger avvike. (...) eldre generasjoner [har] andre forutsetninger for å forstå internett og teknologi, og med det også andre holdninger og forventinger om risikomomenter.''} \parencite[69]{master:hanssen}

\newpage